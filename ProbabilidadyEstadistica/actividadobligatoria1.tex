\hypertarget{funciuxf3n-de-probabilidad}{%
\section{Función de probabilidad}\label{funciuxf3n-de-probabilidad}}

Se denota por \((\Omega, \mathcal{A},P)\) el espacio probabilístico
base. Se considera la siguiente definición de medida de probabilidad:

\textbf{Definición 1.} \(P : \mathcal{A} \rightarrow[0,1]\) es una
función de probabilidad si satisface los siguientes tres axiomas:

\textbf{A1.} \(P(A) \geq 0, \quad \forall A \in \mathcal{A}\)

\textbf{A2.} \(P(\Omega) = 1\)

\textbf{A3.} Para cualquier secuencia
\(\{A_n\}_{n \in \mathbb{N}} \subseteq \mathcal{A}\) de sucesos
disjuntos \[
P\left(\bigcup_{n\in \mathbb{N}}A_n\right) = \sum_{n \in \mathbb{N}} P(A_n)
\]

\hypertarget{propiedades}{%
\section{Propiedades}\label{propiedades}}

\textbf{Demostrar, a partir de la definición anterior, las siguientes
propiedades:}

\begin{itemize}
\item
  \begin{enumerate}
  \def\labelenumi{\alph{enumi})}
  \tightlist
  \item
    \(P(\emptyset) = 0\)
  \end{enumerate}
\end{itemize}

Basta considerar la secuencia de sucesos imposibles
\(\{ A_n \}_{n \in \mathbb{N}}\) con
\(A_n = \emptyset \quad \forall n \in \mathbb{N}\). Es claro que son
disjuntos, luego usando \textbf{A3}

\[
P\left(\bigcup_{n \in \mathbb{N}} \emptyset \right) = \sum_{n \in \mathbb{N}} P(\emptyset)
\]

Observamos que \textbf{A1} garantiza la convergencia de la serie
\(\sum_{n \in \mathbb{N}} P(\emptyset)\) lo cual, de nuevo por
\textbf{A1} se da si y sólo si \(P(\emptyset) = 0\). \(\square\)

\textbf{Nota}: Asumiendo aditividad finita la prueba se deduce con mayor
facilidad: \[
P(\emptyset) = P(\emptyset \cup \emptyset) = P(\emptyset)+ P(\emptyset) 
\]

\begin{itemize}
\item
  \begin{enumerate}
  \def\labelenumi{\alph{enumi})}
  \setcounter{enumi}{1}
  \tightlist
  \item
    Aditividad finita para sucesos disjuntos . Sean
    \(A_1, A_2, \dots, A_N \in \mathcal{A}\) un número finito de sucesos
    disjuntos, entonces \[
    P\left(\bigcup_{n=1}^N A_n \right) = \sum_{n=1}^N P(A_n)
    \]
  \end{enumerate}
\end{itemize}

\textbf{Dem.}

Basta considerar la secuencia extendida \(\{A_n\}_{n \in \mathbb{N}}\)
tal que \(A_n = \emptyset \quad \forall n > N\)

Aplicando \textbf{A3} (aditividad contable). \[
P\left(\bigcup_{n \in \mathbb{N}} A_n \right) = \sum_{n \in \mathbb{N}} P(A_n)
\]

Que en este caso, es equivalente a

\[
P\left(\bigcup_{n=1}^N A_n\right) = \sum_{n=1}^N P(A_n) + \sum_{n=N+1}^\infty P(\emptyset) =  \sum_{n \in \mathbb{N}}^N P(A_n)
\]

utilizando que \(P(\emptyset) = 0.\) \(\square\)

\begin{itemize}
\item
  \begin{enumerate}
  \def\labelenumi{\alph{enumi})}
  \setcounter{enumi}{2}
  \tightlist
  \item
    Probabilidad del suceso complementario. Para cada suceso
    \(A \in \mathcal{A}\) \[
    P(A^c) = 1 - P(A)
    \]
  \end{enumerate}
\end{itemize}

\textbf{Dem.} Aplicando la aditividad finita para \(A\) y su
complementario \(A^c\), que son disjuntos, se tiene:

\[
1 = P(\Omega) = P(A \cup A^c) = P(A) + P(A^c) \\
P(A^c) = 1 - P(A)
\]

Donde se ha utilizado \textbf{A2.} \(\square\)

\textbf{Nota:} sin considerar la aditividad finita es necesario extender
la secuencia con vacíos. Por eso, se ha alterado el orden del enunciado.

\begin{itemize}
\item
  \begin{enumerate}
  \def\labelenumi{\alph{enumi})}
  \setcounter{enumi}{3}
  \tightlist
  \item
    Probabilidad de la diferencia y monotonía. Sea
    \(B \subseteq A \in \mathcal{A}\), entonces \[
    P(A −B) = P(A) −P(B)
    \] y \(P(B) \leq P(A)\)
  \end{enumerate}
\end{itemize}

\textbf{Dem.}

Consideramos los sucesos disjuntos \(B\) y \(A-B\), cuya unión es \(A\)
y aplicamos aditividad finita

\[
P(A) = P(B \cup (A-B)) = P(B) + P(A-B)
\]

Como \(P(A-B) \geq 0\) por \textbf{A1}, deducimos \(P(B) \leq P(A)\). Y
despejando,

\[
P(A-B) = P(A) - P(B)
\]

\(\square\)

\textbf{Nota:} Se obtiene como caso particular la probabilidad del
complementario. \[
P(B^c) = P(\Omega - B) = P(\Omega) - P(B) = 1 - P(B)
\]

\begin{itemize}
\item
  \begin{enumerate}
  \def\labelenumi{\alph{enumi})}
  \setcounter{enumi}{4}
  \tightlist
  \item
    Probabilidad de la unión Sea \(A, B \in \mathcal{A}\), entonces la
    probabilidad de la unión mediante la siguiente fórmula \[
    P(A \cup B) = P(A)+P(B)−P(A \cap B).
    \] En particular, \(P(A\cup B) \leq P(A) + P(B)\)
  \end{enumerate}
\end{itemize}

\textbf{Dem.}

Consideramos los sucesos disjuntos \(A\) y \(B-A\) cuya unión es
\(A \cup B\) y aplicamos aditividad finita

\[
P(A \cup B) = P(A \cup (B-A)) = P(A) + P(B-A)
\]

Claramente \(B-A = B - (A \cap B)\) con \(A \cap B \subset B\) luego

\[
P(A \cup B) = P(A)  + P(B - (A \cap B)) = P(A) + P(B) - P(A \cap B)
\]

Donde se ha aplicado la probabilidad de la diferencia. \(\square\)

\begin{itemize}
\item
  \begin{enumerate}
  \def\labelenumi{\alph{enumi})}
  \setcounter{enumi}{5}
  \tightlist
  \item
    Principio de inclusión-exclusión para la unión finita de sucesos no
    disjuntos. Sea \(A_1, A_2, \dots, A_n \in \mathcal{A}\), entonces
  \end{enumerate}
\end{itemize}

\[
P\left(\bigcup_{i=1}^n A_i \right) = \sum_{i=1}^n P(A_i) - \sum_{i < j}^n P(A_i \cap A_j) + \sum_{i < j < k}^n P(A_i \cap A_j \cap A_k) + \dots  + (-1)^{n-1} P\left(\bigcap_{i=1}^n A_i \right)
\]

\textbf{Dem.}

Por inducción.

\begin{itemize}
\item
  \textbf{Caso base \(n=1\):} Trivial \[
  P\left(\bigcup_{i=1}^1 A_i \right) = \sum_{i=1}^1 P(A_i) = P(A_1)
  \]
\item
  \textbf{Caso \(n=2\)}:
\end{itemize}

Es la fórmula de la probabilidad de la unión de dos sucesos demostrada
anteriormente.

\[
P(A \cup B) = P(A) + P(B) - P(A \cap B)
\]

\begin{itemize}
\tightlist
\item
  \textbf{Paso inductivo}: suponiendo que el resultado es cierto para
  \(n\), lo demotramos para \(n+1\).
\end{itemize}

Considerando los sucesos disjuntos \(\bigcup_{i=1}^n A_i\) y
\((A_{n+1} - \bigcup_{i=1}^n A_i)\) cuya unión es
\(\bigcup_{i=1}^{n+1} A_i\) aplicamos aditividad finita

\[
P\left(\bigcup_{i=1}^{n+1} A_i \right) = P\left(\bigcup_{i=1}^n A_i \cup \left(A_{n+1} - \bigcup_{i=1}^n A_i\right) \right) = 
P\left(\bigcup_{i=1}^n A_i \right) + P\left(A_{n+1} - \bigcup_{i=1}^n A_i \right)
\]

En general notamos que
\(P(A-B) = P(A - (A \cap B)) = P(A) - P(A\cap B)\) usando la
probabilidad de la diferencia. Lo aplicamos al último sumando del
mimebro derecho.

\[
P\left(\bigcup_{i=1}^{n+1} A_i \right) = 
P\left(\bigcup_{i=1}^n A_i \right) + P\left(A_{n+1} - \bigcup_{i=1}^n A_i \right) =
P\left(\bigcup_{i=1}^n A_i \right) + P\left(A_{n+1} \right) - P\left(\bigcup_{i=1}^n A_i \cap A_{n+1} \right)
\]

Aplicando la hipótesis de inducción al primer y tercer sumando del
segundo miembro:

\[
P\left(\bigcup_{i=1}^{n+1} A_i \right) = 
\sum_{i=1}^n P(A_i) - \sum_{i < j}^n P(A_i \cap A_j) + \dots  + (-1)^{n-1} P\left(\bigcap_{i=1}^n A_i \right) + P(A_{n+1}) \\ - \sum_{i=1}^n P(A_i \cap A_{n+1}) + \sum_{i < j}^n P(A_i \cap A_j \cap A_{n+1}) - \dots  + \\ + (-1)^{n-1}\sum_{a_1 < a_2 < ... < a_{n - 1}}^n P\left(A_{a_1} \cap \dots \cap  A_{a_{n-1}} \cap A_{n+1}\right)- (-1)^{n-1} P\left(\bigcap_{i=1}^n A_i \cap A_{n+1} \right) 
\]

Agrupamos los términos, sumando el \(i\)-ésimo con el \((n+i)\)-ésimo.

\textbf{a) Primer término con el \((n+1)\)-ésimo}

\[ \sum_{i=1}^n P(A_i) + P(A_{n+1}) = \sum_{i=1}^{n+1} P(A_i) \]

\textbf{b) Segundo término con \((n+2)\)-ésimo\$}

\[-\sum_{i < j}^n P(A_i \cap A_j) - \sum_{i=1}^n P(A_i \cap A_{n+1}) = -\sum_{i < j}^{n+1} P(A_i \cap A_j)\]

\textbf{c) Tercer término con \((n+3)\)-ésimo}

\[\sum_{i < j < k}^n P(A_i \cap A_j \cap A_k) + \sum_{i < j}^n P(A_i \cap A_j \cap A_{n+1}) = \sum_{i < j < k} ^{n+1} P(A_i \cap A_j \cap A_k)\]

\textbf{\(\dots\)}

\textbf{d) \(n\)-ésimo con \((2n)\)-ésimo término}

\[
(-1)^{n-1}P\left(\bigcap_{i=1}^n A_i \right) + (-1)^{n-1}\sum_{a_1 < a_2 < ... < a_{n - 1}}^n P\left(A_{a_1} \cap \dots \cap  A_{a_{n-1}} \cap A_{n+1}\right) = \sum_{a_1 < a_2 < \dots a_n}^{n+1} P\left(\bigcap_{i=1}^n A_{a_i}\right)
\]

Finalmente, notamos que \(-(-1)^{n-1} = (-1)^n\) y que
\(\bigcap_{i=1}^n A_i \cap A_{n+1} = \bigcap_{i=1}^{n+1} A_i\) en el
último término luego

\[
P\left(\bigcup_{i=1}^{n+1} A_i \right) = \sum_{i=1}^{n+1} P(A_i) - \sum_{i < j}^{n+1} P(A_i \cap A_j) + \sum_{i < j < k}^{n+1} P(A_i \cap A_j \cap A_k) + \dots  + (-1)^nP\left(\bigcap_{i=1}^{n+1} A_i \right)
\]

\(\square\)

\begin{itemize}
\item
  \begin{enumerate}
  \def\labelenumi{\alph{enumi})}
  \setcounter{enumi}{6}
  \tightlist
  \item
    Subaditividad: \[
    P\left(\bigcup_{n \in \mathbb{N}} A_n\right) \leq \sum_{n \in \mathbb{N}} P(A_n).
    \]
  \end{enumerate}
\end{itemize}

Definimos una nueva familia de sucesos:

\[
B_1 = A_1 \\
B_2 = A_2 - A_1 \\
B_3 = A_3 - (A_1 \cup A_2)
\dots \\
B_n = A_n - \bigcup_{j=1}^{n-1} A_n
\]

Observamos lo siguiente:

\[B_i \cap B_j = \emptyset\]
\[\bigcup_{n \in \mathbb{N}} A_n = \bigcup_{n \in \mathbb{N}} B_n \]

Luego aplicamos \textbf{A3} a la unión:

\[
P\left(\bigcup_{n \in \mathbb{N}} A_i \right) = P\left(\bigcup_{n \in \mathbb{N}} B_i \right) = \sum_{n \in \mathbb{N}} P(B_i)
\]

Además, como \(B_n \subset A_n \quad \forall n \in \mathbb{N}\), usamos
la consecuencia de la probabilidad de la diferencia
\(P(B_n) \leq P(A_n)\)

y tenemos que

\[
\sum_{i=1}^n P(B_i) \leq \sum_{i=1}^n P(A_i)
\]

Luego, eventualmente en el límite:

\[
\sum_{n \in \mathbb{N}} P(B_i) \leq \sum_{n \in \mathbb{N}} P(A_i)
\]

y llegamos al resultado

\[
P\left(\bigcup_{n \in \mathbb{N}} A_n\right) \leq \sum_{n \in \mathbb{N}} P(A_n).
\]

\(\square\)

\begin{itemize}
\item
  \begin{enumerate}
  \def\labelenumi{\alph{enumi})}
  \setcounter{enumi}{7}
  \tightlist
  \item
    Desigualdad de Boole: \[
    P\left(\bigcap_{n \in \mathbb{N}} A_n \right) \geq 1 − \sum_{n \in \mathbb{N}} P(A_n^c)
    \]
  \end{enumerate}
\end{itemize}

Usamos que
\(\bigcap_{n \in \mathbb{N}} A_n = \left(\bigcup_{n \in \mathbb{N}} A_n^c \right)^c\)
y la probabilidad del suceso complementario

\[
P\left( \bigcap_{n \in \mathbb{N}}A_n \right) = 
P\left( \left(\bigcup_{n \in \mathbb{N}} A_n^c \right)^c \right) = 
1 - P\left( \bigcup_{n \in \mathbb{N}}A_n^c \right)
\]

Utilizando la propiedad de subadtividad

\[
  P\left(\bigcap_{n \in \mathbb{N}} A_n \right) \geq 1 − \sum_{n \in \mathbb{N}} P(A_n^c)
\]

\(\square\)
