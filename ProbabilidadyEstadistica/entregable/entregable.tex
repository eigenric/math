\documentclass[
  a4paper,
  spanish,
  12pt,
]{scrartcl}

\linespread{1.05}
%\setlength{\parindent}{18pt}


%-------------------------------------------------------------------------------
%	PAQUETES
%-------------------------------------------------------------------------------

% Idioma

\usepackage[es-noindentfirst, es-tabla]{babel}

% Matemáticas

\usepackage{config}
% \usepackage{amsmath, amsthm, amssymb}
% \usepackage{mathtools}
% \usepackage{commath}
% \usepackage{xfrac}


% Fuentes personalizadas para utilizar con XeLaTeX o LuaLaTeX

\usepackage[no-math]{fontspec}
\setmainfont[WordSpace=1.3, RawFeature={+ss06}]{EBGaramond}
\setsansfont[Scale=0.9]{Alegreya Sans}
\setmonofont[Scale=0.75]{Vera Mono}

\usepackage[math-style=TeX]{unicode-math}
\setmathfont{Garamond Math}[StylisticSet={3}]


% Configuración de microtype

\defaultfontfeatures{Ligatures=TeX,Numbers=Lining}
\usepackage[activate={true,nocompatibility},final,tracking=true,factor=1100,stretch=10,shrink=10]{microtype}
\SetTracking{encoding={*}, shape=sc}{0}

% Enlaces y colores

\usepackage{hyperref}
\usepackage{xcolor}
\hypersetup{
  colorlinks=true,
  citecolor=,
  linkcolor=,
  urlcolor=blue,
}

% Otros elementos de página

\usepackage{enumitem}
\setlist[enumerate]{leftmargin=-\itemindent, itemsep=0pt}
\setlist[itemize]{leftmargin=-\itemindent, itemsep=0pt}
%\setlist[itemize]{leftmargin=*}
%\setlist[enumerate]{leftmargin=*}

\usepackage[labelfont={sc, sf}, textfont=sf]{caption}

\usepackage{booktabs}
\renewcommand\arraystretch{1.5}

% Tikz

\usepackage{tikz}
\usetikzlibrary{babel}
\usepackage{float}

% Código

\usepackage{listings}
\lstset{
	basicstyle=\ttfamily,%
	breaklines=true,%
	captionpos=b,                    % sets the caption-position to bottom
  tabsize=2,	                   % sets default tabsize to 2 spaces
  frame=lines,
  numbers=left,
  stepnumber=1,
  aboveskip=12pt,
  showstringspaces=false,
  keywordstyle=\bfseries,
  commentstyle=\itshape,
  columns=flexible,
}
\renewcommand{\lstlistingname}{Listado}

% ENTORNOS

\usepackage[theorems, skins, breakable]{tcolorbox}

\tcolorboxenvironment{nth}{
	blanker,
	breakable,
	left=12pt,
	before skip=12pt,
	after skip=12pt,
	borderline west={2pt}{0pt}{500},
	before upper={\parindent 12pt},
}

\tcolorboxenvironment{nprop}{
	blanker,
	breakable,
	left=12pt,
	before skip=12pt,
	after skip=12pt,
	borderline west={2pt}{0pt}{42},
	before upper={\parindent 12pt},
}

\tcolorboxenvironment{ncor}{
	blanker,
	breakable,
	left=12pt,
	before skip=12pt,
	after skip=12pt,
	borderline west={2pt}{0pt}{300},
	before upper={\parindent 12pt},
}

\tcolorboxenvironment{ndef}{
	skin=enhancedmiddle jigsaw,
	frame hidden,
	colback= 36,
	breakable = true,
	break at = -6pt,
	top = 4pt,       % Estos márgenes están un poco a ojo
	bottom = 4pt,
	left= 8pt,
	right = 8pt,
	before skip=8pt, % Normalmente dejamos 12pt, pero
	after skip=8pt,  % aquí tenemos espacio adicional por el fondo
	no borderline,
	borderline west={2pt}{0pt}{42},
	before upper={\parindent 12pt},
}

\tcolorboxenvironment{ejer}{
	skin=enhancedmiddle jigsaw,
	frame hidden,
	colback=50,
	breakable = true,
	break at = -6pt,
	top = 4pt,       % Estos márgenes están un poco a ojo
	bottom = 4pt,
	left= 8pt,
	right = 8pt,
	before skip=8pt, % Normalmente dejamos 12pt, pero
	after skip=8pt,  % aquí tenemos espacio adicional por el fondo
	no borderline,
	borderline west={2pt}{0pt}{500},
	borderline east={2pt}{0pt}{50},
	before upper={\parindent 12pt},
}


% Márgenes
\usepackage[bottom=3.125cm, top=2.5cm, left=3.5cm, right=3.5cm, marginparwidth=70pt]{geometry}

%dealing with (sub/subsub)sections
%\let\raggedsection\centering%Center all sectioningheads
%all levels have something in common, let's save typing:
\RedeclareSectionCommands[beforeskip=-3ex,
afterskip=2ex]{section,subsection,subsubsection}
\addtokomafont{section}{\normalfont\large\textsc}
\RedeclareSectionCommand[beforeskip=-6ex]{section}
\addtokomafont{subsection}{\normalfont\itshape}

%-------------------------------------------------------------------------------
%	CONTENIDO
%-------------------------------------------------------------------------------

\begin{document}

\begin{flushright}
  Ricardo Ruiz Fernández de Alba\vspace{.5em}

  \textit{Probabilidad}

  Doble Grado en Ingeniería Informática y Matemáticas

  \textsc{Universidad de Granada}\vspace{.5em}

  \today\vspace{.5em}
\end{flushright}

\begin{flushleft}
  \scshape\Large Actividad Evaluable 1.
\end{flushleft}

\section{Función de Probabilidad}

Se denota por $(\Omega, \mathcal{A},P)$ el espacio probabilístico base. Se considera la siguiente definición de medida de probabilidad:

\begin{ndef}
   $P : \mathcal{A} \rightarrow[0,1]$ es una función de probabilidad si satisface los siguientes tres axiomas:

\emph{A1.} $P(A) \geq 0, \quad \forall A \in \mathcal{A}$

\emph{A2.} $P(\Omega) = 1$

\emph{A3.} Para cualquier secuencia $\{A_n\}_{n \in \mathbb{N}} \subseteq \mathcal{A}$ de sucesos disjuntos
$$
P\left(\bigcup_{n\in \mathbb{N}}A_n\right) = \sum_{n \in \mathbb{N}} P(A_n)
$$
\end{ndef}


\section{Propiedades}

Demostrar, a partir de la definición anterior, las siguientes propiedades

\begin{nprop}
    $P(\emptyset) = 0$
\end{nprop}

\begin{proof}

Basta considerar la secuencia de sucesos imposibles $\{ A_n \}_{n \in \mathbb{N}}$ con $A_n = \emptyset \quad \forall n \in \mathbb{N}$. Es claro que son disjuntos, luego usando \emph{A3}

$$
P\left(\bigcup_{n \in \mathbb{N}} \emptyset \right) = \sum_{n \in \mathbb{N}} P(\emptyset)
$$

Observamos que \emph{A1} garantiza la convergencia de la serie $\sum_{n \in \mathbb{N}} P(\emptyset)$ lo cual, de nuevo por \emph{A1} se da si y sólo si $P(\emptyset) = 0$.

\end{proof}

\begin{nota}
Asumiendo aditividad finita la prueba se deduce con mayor facilidad:
$$
P(\emptyset) = P(\emptyset \cup \emptyset) = P(\emptyset)+ P(\emptyset) 
$$
\end{nota}

\begin{nprop}
Aditividad finita para sucesos disjuntos.

Sean $A_1, A_2, \dots, A_N \in \mathcal{A}$ un número finito de sucesos disjuntos, entonces
$$
P\left(\bigcup_{n=1}^N A_n \right) = \sum_{n=1}^N P(A_n)
$$
\end{nprop}

\begin{proof}

Basta considerar la secuencia extendida $\{A_n\}_{n \in \mathbb{N}}$ tal que 
$A_n = \emptyset \quad \forall n > N$ 

Aplicando \emph{A3} (aditividad contable).
$$
P\left(\bigcup_{n \in \mathbb{N}} A_n \right) = \sum_{n \in \mathbb{N}} P(A_n)
$$

Que en este caso, es equivalente a 

$$
P\left(\bigcup_{n=1}^N A_n\right) = \sum_{n=1}^N P(A_n) + \sum_{n=N+1}^\infty P(\emptyset) =  \sum_{n \in \mathbb{N}}^N P(A_n)
$$

utilizando que $P(\emptyset) = 0.$ 

\end{proof}

\begin{nprop}
 Probabilidad del suceso complementario. Para cada suceso $A \in \mathcal{A}$
$$
P(A^c) = 1 - P(A)
$$
\end{nprop}

\begin{proof}
Aplicando la aditividad finita para $A$ y su complementario $A^c$, que son disjuntos, se tiene:

$$
1 = P(\Omega) = P(A \cup A^c) = P(A) + P(A^c)
$$
Donde se ha utilizado \emph{A2}. Luego,

$$
P(A^c) = 1 - P(A)
$$

\end{proof}

\begin{nota}
Sin considerar la aditividad finita es necesario extender la secuencia con vacíos. Por eso, se ha alterado el orden del enunciado.
\end{nota}

\begin{nprop}
     Probabilidad de la diferencia y monotonía. Sea $B \subseteq A \in \mathcal{A}$, entonces
$$
P(A-B) = P(A)-P(B)
$$ 
y $P(B) \leq P(A)$
\end{nprop}

\begin{proof}
Consideramos los sucesos disjuntos $B$ y $A-B$, cuya unión es $A$ y aplicamos aditividad finita

$$
P(A) = P(B \cup (A-B)) = P(B) + P(A-B)
$$

Como $P(A-B) \geq 0$ por \emph{A1}, deducimos $P(B) \leq P(A)$. 
Y despejando,

$$
P(A-B) = P(A) - P(B)
$$

\end{proof}

\begin{nota}
    Como caso particular se obtiene la probabilidad del complementario.
$$
P(B^c) = P(\Omega - B) = P(\Omega) - P(B) = 1 - P(B)
$$
\end{nota}

\begin{nprop}
Probabilidad de la unión.

Sean $A, B \in \mathcal{A}$, entonces la probabilidad de la unión se halla mediante la siguiente fórmula

$$
P(A \cup B) = P(A) + P(B) - P(A \cap B)
$$

En particular, $P(A\cup B) \leq P(A) + P(B)$
\end{nprop}

\begin{proof}

Consideramos los sucesos disjuntos $A$ y $B-A$ cuya unión es $A \cup B$ y aplicamos aditividad finita

$$
P(A \cup B) = P(A \cup (B-A)) = P(A) + P(B-A)
$$

Claramente $B-A = B - (A \cap B)$ con $A \cap B \subset B$ luego

$$
P(A \cup B) = P(A)  + P(B - (A \cap B)) = P(A) + P(B) - P(A \cap B)
$$

Donde se ha aplicado la probabilidad de la diferencia. 

\end{proof}

\begin{nprop}
Principio de inclusión-exclusión para la unión finita de sucesos no disjuntos. Sean $A_1, A_2, \dots, A_n \in \mathcal{A}$, entonces

\begin{equation*}
\begin{aligned}
P\left(\bigcup_{i=1}^n A_i \right) = \sum_{i=1}^n P(A_i) - \sum_{i < j}^n P(A_i \cap A_j) + \sum_{i < j < k}^n P(A_i \cap A_j \cap A_k) + \dots \\
+ (-1)^{n-1} P\left(\bigcap_{i=1}^n A_i \right)
\end{aligned}
\end{equation*}

\end{nprop}

\begin{proof}

Por inducción.

- \emph{Caso base $n=1$:} Trivial

$$
P\left(\bigcup_{i=1}^1 A_i \right) = \sum_{i=1}^1 P(A_i) = P(A_1)
$$

- \emph{Caso $n=2$:}

Es la fórmula de la probabilidad de la unión de dos sucesos demostrada anteriormente.

$$
P(A \cup B) = P(A) + P(B) - P(A \cap B)
$$

- \emph{Paso inductivo:} suponiendo que el resultado es cierto para $n$, lo demostramos para $n+1$.

Considerando los sucesos disjuntos $\bigcup_{i=1}^n A_i$ y $(A_{n+1} - \bigcup_{i=1}^n A_i)$ cuya unión es $\bigcup_{i=1}^{n+1} A_i$ aplicamos aditividad finita

\begin{equation*}
\begin{aligned}
P\left(\bigcup_{i=1}^{n+1} A_i \right) = P\left(\bigcup_{i=1}^n A_i \cup \left(A_{n+1} - \bigcup_{i=1}^n A_i\right) \right) = 
P\left(\bigcup_{i=1}^n A_i \right) + P\left(A_{n+1} - \bigcup_{i=1}^n A_i \right)
\end{aligned}
\end{equation*}

En general notamos que $P(A-B) = P(A - (A \cap B)) = P(A) - P(A\cap B)$ usando la probabilidad de la diferencia. Lo aplicamos al último sumando del mimebro derecho.

\begin{equation*}
\begin{aligned}
P\left(\bigcup_{i=1}^{n+1} A_i \right) = 
P\left(\bigcup_{i=1}^n A_i \right) + P\left(A_{n+1} - \bigcup_{i=1}^n A_i \right) = \\
=  P\left(\bigcup_{i=1}^n A_i \right) + P\left(A_{n+1} \right) - P\left(\bigcup_{i=1}^n A_i \cap A_{n+1} \right)
\end{aligned}
\end{equation*}

Aplicando la hipótesis de inducción al primer y tercer sumando del miembro de la derecha:

\begin{equation*}
\begin{aligned}
P\left(\bigcup_{i=1}^{n+1} A_i \right) = 
\sum_{i=1}^n P(A_i) - \sum_{i < j}^n P(A_i \cap A_j) + \dots  + (-1)^{n-1} P\left(\bigcap_{i=1}^n A_i \right) + P(A_{n+1}) +\\ 
- \sum_{i=1}^n P(A_i \cap A_{n+1}) + \sum_{i < j}^n P(A_i \cap A_j \cap A_{n+1}) - \dots  + \\
+ (-1)^{n-1}\sum_{a_1 < a_2 < ... < a_{n - 1}}^n P\left(A_{a_1} \cap \dots \cap  A_{a_{n-1}} \cap A_{n+1}\right)
\\ - (-1)^{n-1} P\left(\bigcap_{i=1}^n A_i \cap A_{n+1} \right) 
\end{aligned}
\end{equation*}

Agrupamos los términos, sumando el $i$-ésimo con el $(n+i)$-ésimo.

\emph{Primer término con el $(n+1)$-ésimo}

$$ \sum_{i=1}^n P(A_i) + P(A_{n+1}) = \sum_{i=1}^{n+1} P(A_i) $$

\emph{Segundo término con $(n+2)$-ésimo}

$$-\sum_{i < j}^n P(A_i \cap A_j) - \sum_{i=1}^n P(A_i \cap A_{n+1}) = -\sum_{i < j}^{n+1} P(A_i \cap A_j)$$

\emph{Tercer término con $(n+3)$-ésimo}

$$\sum_{i < j < k}^n P(A_i \cap A_j \cap A_k) + \sum_{i < j}^n P(A_i \cap A_j \cap A_{n+1}) = \sum_{i < j < k} ^{n+1} P(A_i \cap A_j \cap A_k)$$

$\dots$

\emph{$n$-ésimo con $(2n)$-ésimo término}

$$
(-1)^{n-1}P\left(\bigcap_{i=1}^n A_i \right) + (-1)^{n-1}\sum_{a_1 < a_2 < ... < a_{n - 1}}^n P\left(A_{a_1} \cap \dots \cap  A_{a_{n-1}} \cap A_{n+1}\right) = \sum_{a_1 < a_2 < \dots a_n}^{n+1} P\left(\bigcap_{i=1}^n A_{a_i}\right)
$$


Finalmente, notamos que $-(-1)^{n-1} = (-1)^n$ y que $\bigcap_{i=1}^n A_i \cap A_{n+1} = \bigcap_{i=1}^{n+1} A_i$ en el último término luego

$$
P\left(\bigcup_{i=1}^{n+1} A_i \right) = \sum_{i=1}^{n+1} P(A_i) - \sum_{i < j}^{n+1} P(A_i \cap A_j) + \sum_{i < j < k}^{n+1} P(A_i \cap A_j \cap A_k) + \dots  + (-1)^nP\left(\bigcap_{i=1}^{n+1} A_i \right)
$$

\end{proof}

\begin{nprop}
     Subaditividad: 
$$
P\left(\bigcup_{n \in \mathbb{N}} A_n\right) \leq \sum_{n \in \mathbb{N}} P(A_n).
$$
\end{nprop}

\begin{proof}
Definimos una nueva familia de sucesos:

\begin{equation*}
\begin{aligned}
B_1 = A_1 \\
B_2 = A_2 - A_1 \\
B_3 = A_3 - (A_1 \cup A_2) \\
\dots \\
B_n = A_n - \bigcup_{j=1}^{n-1} A_n
\end{aligned}
\end{equation*}

Y observamos que son disjuntos dos a dos y que la unión de esta familia coincide con la de $A_n$:

$$B_i \cap B_j = \emptyset$$
$$\bigcup_{n \in \mathbb{N}} A_n = \bigcup_{n \in \mathbb{N}} B_n$$

Luego aplicamos \emph{A3} a la unión:

$$
P\left(\bigcup_{n \in \mathbb{N}} A_i \right) = P\left(\bigcup_{n \in \mathbb{N}} B_i \right) = \sum_{n \in \mathbb{N}} P(B_i)
$$

Además, como $B_n \subset A_n \quad \forall n \in \mathbb{N}$, usamos la monotonía de la probabilidad $P(B_n) \leq P(A_n)$ y tenemos que 

$$
\sum_{i=1}^n P(B_i) \leq \sum_{i=1}^n P(A_i)
$$

Por tanto, eventualmente en el límite:

$$
\sum_{n \in \mathbb{N}} P(B_i) \leq \sum_{n \in \mathbb{N}} P(A_i)
$$

llegando al resultado

$$
P\left(\bigcup_{n \in \mathbb{N}} A_n\right) \leq \sum_{n \in \mathbb{N}} P(A_n).
$$

\end{proof}

\begin{nprop}
Desigualdad de Boole: 
$$
P\left(\bigcap_{n \in \mathbb{N}} A_n \right) \geq 1 - \sum_{n \in \mathbb{N}} P(A_n^c)
$$
\end{nprop}

\begin{proof}

Usamos que $\bigcap_{n \in \mathbb{N}} A_n = \left(\bigcup_{n \in \mathbb{N}} A_n^c \right)^c$ y la probabilidad del suceso complementario

$$
P\left( \bigcap_{n \in \mathbb{N}}A_n \right) = 
P\left( \left(\bigcup_{n \in \mathbb{N}} A_n^c \right)^c \right) = 
1 - P\left( \bigcup_{n \in \mathbb{N}}A_n^c \right)
$$

Finalmente, utilizando la propiedad de subaditividad

$$
  P\left(\bigcap_{n \in \mathbb{N}} A_n \right) \geq 1 - \sum_{n \in \mathbb{N}} P(A_n^c)
$$
  
\end{proof}

\end{document}
